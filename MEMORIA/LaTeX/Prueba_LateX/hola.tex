\documentclass[10pt,a4paper]{article}
\usepackage[utf8]{inputenc}
\usepackage[spanish]{babel}
\usepackage{amsmath}
\usepackage{amsfonts}
\usepackage{amssymb}
\usepackage{graphicx}
\usepackage{fourier}
\usepackage[left=2cm,right=2cm,top=2cm,bottom=2cm]{geometry}
\author{Sergy}
\title{Mi primera Prueba de LaTeX }
\begin{document}
\maketitle
\section{Introducción}
Hola que carajo pasa aquí, espero que se ejecute correctamente y se visualice de una manera apropiada. QUIERO OIAUSCVHQWUVYCBQOWUVBHQWUOBVYOQWUBHVOWUVBH.
hOLA esto es punto y aparte.

Pongo sangría?
\subsection{Hola}
A ver esto cómo sale
\section{Parte 1}
\[(a + b)^2\]
Hola que tal.

\textbf{Ahora voy a poner una ecuación numeradaaaaa}
\begin{equation}
a + b / 2
\end{equation}
Voy a escribir un a fracción con el comando $\dfrac{345}{a + b^2}$
\bigskip
Ahora voy a darle a la sumatoria lokooo:

\[\Sigma_{n_1 = 2}^{\infty} \arccos(0.76)\]
\bigskip
\begin{itemize}
\item ¿Ahora cómo lo pone?
\item lista 2
\item lista 3
\item enga otro
\bigskip 
\end{itemize}

Voy a hacer un sistema de ecuaciones (ALINEADO CON EL ''=''). Le hemos puesto el ''*'' para que no se numeren.
\begin{align*}
a + b + y &= 78.54 \\
a - y &= 0
\end{align*}
\bigskip
\textbf{Las comillas se tienen que poner como: ''comillas''}

Esto \\ son saltos \\ de \\ línea

Voy a probar ahora con el comando ''sum'':
\begin{equation}
\sum_{n = 2}^{\infty}3^n
\end{equation}

\textbf{Vemos que es mejor con el comando ''sum''.}

\textit{Esto va a estar en cursiva (o eso espero).}
\underline{Esto tendría que estar subrayado.}

\pagebreak
Ahora procedemos a insertar una imagen, a ver cómo salen:
\begin{figure}[hbtp]
\centering
\includegraphics[width = 7cm]{../../../../../Downloads/fondoapp.jpg}
\caption{Fondo App}
\end{figure}

\section{Tablas}
\begin{center}
\begin{tabular}{|c|c|c|c|}
\hline 
56 & • & • & • \\ 
\hline 
• & hola & • & 89 \\ 
\hline 
• & • & • & • \\ 
\hline 
• & esto se supone que es largo & • & 56 \\ 
\hline 
\end{tabular} 
\end{center}

\begin{center}
\begin{tabular}{c|c|c|c}
\hline 
• & • & • & • \\ 
\hline 
• & • & • & • \\ 
\hline 
• & • & • & • \\ 
\hline 
• & • & • & • \\ 
\hline 
• & • & • & • \\ 
\hline 
\end{tabular}\bigskip

Tabla 1: Datos
\end{center}



\end{document}