\chapter*{}
%\thispagestyle{empty}
%\cleardoublepage

%\thispagestyle{empty}

\input{portada/portada_2}



\cleardoublepage
\thispagestyle{empty}

\begin{center}
{\large\bfseries Beamformer acústico: Implementación de un Beamformer acústico mediante un array Ad-Hoc de smartphones.}\\
\end{center}
\begin{center}
Sergio Zapata Caparrós\\
\end{center}

\vspace{0.7cm}
\noindent{\textbf{Palabras clave}: beamforming, beamformer, sincronización, sincronismo, array, directividad, chirp, impulso, correlación, ruido, retardo, servidor, dispositivos, micrófonos, frecuencia, orientación, señal, distancia}\\

\vspace{0.7cm}
\noindent{\textbf{Resumen}}\\

Numerosas situaciones en las que se desea realizar un procesamiento de señales acústicas, requieren de una buena sincronización entre señales, con el objetivo de cometer un error mínimo. Entre estas situaciones, se encuentran por ejemplo la localización de fuentes sonoras o el proceso de beamforming. 

En este proyecto se ha propuesto un método de sincronización alternativo, estudiado en un array lineal de smartphones, con el propósito de poder llegar a implementar correctamente un beamformer y obtener finalmente el patrón de directividad experimental resultante del array.

Como primera instancia, se proporcionará unas nociones teóricas relacionadas con el procesamiento acústico de señales. El proyecto continúa con la creación de una aplicación, que actuará como cliente en cada dispositivo móvil, y un servidor, siendo dos herramientas necesarias para la correcta comunicación de todos los elementos del array y las cuales posibilitan el posterior tratamiento de las señales grabadas por los smartphones.

Con una comunicación app-servidor fiable, se conmutará al desarrollo del método de sincronización y, para concluir, se comprobará el funcionamiento de un beamformer Delay \& Sum para un array lineal de un total de tres dispositivos móviles.
\cleardoublepage


\thispagestyle{empty}


\begin{center}
{\large\bfseries Acoustic Beamformer: Implementation of an acoustic Beamformer using an Ad-Hoc array of smartphones}\\
\end{center}
\begin{center}
Sergio Zapata Caparrós\\
\end{center}

\vspace{0.7cm}
\noindent{\textbf{Keywords}: beamforming, beamformer, synchronization, 
synchronism, array, directivity, chirp, impulse, correlation, noise, delay, server, devices, microphones, frequency, orientation, signal, distance }\\

\vspace{0.7cm}
\noindent{\textbf{Abstract}}\\

Several situations in which an acoustic signal processing is desired, an accurate synchronization between signals is required, in order to obtain the minimum error. These situations are, for instance, the location of sound sources or a beamforming process.

In this paper, an alternative synchronization method has been proposed. This method has been studied in a linear array of Android smartphones, with the aim of being able to correctly implement a beamformer and, as a result, obtain the experimental directivity pattern from the array.

First of all, some theoretical notions related to acoustic signal processing will be provided. The project continues with the creation of an app, which will play the role of a client on each smartphone, and a server, being these tools necessary for the correct communication between all the elements of the array.

Once a reliable app-server communication is achieved, the synchronization method will be developed and, eventually, the behaviour of a Delay \& Sum beamformer for a linear array of three devices will be verified.


\chapter*{}
\thispagestyle{empty}

\noindent\rule[-1ex]{\textwidth}{2pt}\\[4.5ex]

Yo, \textbf{Sergio Zapata Caparrós}, alumno de la titulación INGENIERÍA DE TECNOLOGÍAS DE TELECOMUNICACIÓN de la \textbf{Escuela Técnica Superior
de Ingenierías Informática y de Telecomunicación de la Universidad de Granada}, con DNI 77145800M, autorizo la
ubicación de la siguiente copia de mi Trabajo Fin de Grado en la biblioteca del centro para que pueda ser
consultada por las personas que lo deseen.

\vspace{6cm}

\noindent Fdo: Sergio Zapata Caoarrós

\vspace{2cm}

\begin{flushright}
Granada a 8 de Julio de 2022.
\end{flushright}


\chapter*{}
\thispagestyle{empty}

\noindent\rule[-1ex]{\textwidth}{2pt}\\[4.5ex]

D. \textbf{Antonio Miguel Peinado Herreros}, Profesor del Área de Teoría de la Señal del Departamento de Teoría de la Señal, Telemática y Comunicaciones de la Universidad de Granada.

\vspace{0.5cm}

D. \textbf{Nombre Ángel Manuel Gómez García}, Profesor del Área de Teoría de la Señal del Departamento de Teoría de la Señal, Telemática y Comunicaciones de la Universidad de Granada.


\vspace{0.5cm}

\textbf{Informan:}

\vspace{0.5cm}

Que el presente trabajo, titulado \textit{\textbf{Beamforming acústico}},
ha sido realizado bajo su supervisión por \textbf{Sergio Zapata Caparrós}, y autorizamos la defensa de dicho trabajo ante el tribunal
que corresponda.

\vspace{0.5cm}

Y para que conste, expiden y firman el presente informe en Granada a 8 de Julio de 2022.

\vspace{1cm}

\textbf{Los directores:}

\vspace{5cm}

\noindent \textbf{Antonio Miguel Peinado Herreros \ \ \ \ \ Ángel Manuel Gómez García}

\chapter*{Agradecimientos}
\thispagestyle{empty}

       \vspace{1cm}


Primeramente, me gustaría agradecer a mis tutores Antonio y Ángel, los cuales me han facilitado mucho la realización del proyecto, resolviéndome diversos tipos de dudas y proponiéndome un abanico de alternativas cuando algún problema surgía. Han sido unos directores de trabajo excelentes.

       \vspace{1cm}

A mi familia, que siempre ha estado apoyándome en los cuatro años de carrera y en el desarrollo del trabajo, facilitándome algunos materiales necesarios para pruebas experimentales y animándome día a día.

       \vspace{1cm}

También quisiera agradecer tanto a la universidad de Granada como a los profesores que han contribuido a mi formación y me han llenado de ganas y curiosidad por el ámbito de las telecomunicaciones.


